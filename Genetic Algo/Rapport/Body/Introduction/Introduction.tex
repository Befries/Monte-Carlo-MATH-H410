\section{Introduction}

Genetic algorithms (GAs) are a class of computational techniques inspired by 
the principles of natural selection and genetic processes. 
They are designed to identify optimal or near-optimal solutions to
complex problems by mimicking evolutionary mechanisms.

The process begins with the initialization of a random population,
where each individual represents a potential solution to the problem at hand. 
The characteristics of these individuals are encoded within a structure akin to chromosomes, 
which define their genetic makeup. The quality or fitness of each individual is then assessed using 
a fitness function, which evaluates the effectiveness of the corresponding solution.

As in natural evolution, a selection process ensures that only the fittest individuals
contribute to the next generation. These individuals reproduce by combining their genetic material 
through a mechanism known as chromosome crossover. During this process, 
the genetic information from two parents is randomly combined to form the genetic code of their 
offspring.

To maintain diversity within the population and prevent premature convergence to suboptimal solutions,
random mutations are introduced into the genetic code of new individuals.
This variation helps explore the solution space more effectively.

The offspring created through crossover and mutation constitute the new generation, 
which undergoes the same cycle of evaluation, selection, and reproduction.
This iterative process continues until a solution meeting the desired level of optimization is achieved.

Genetic algorithms have found widespread applications across various fields. In mechanical engineering,
they are often integrated into computer-aided design (CAD) software to optimize designs based on multiple
performance parameters. 

In the industrial sector, GAs are utilized to optimize production plans while accounting for dynamic
 factors such as inventory levels and material quality. 

In the biomedical field, GAs play a significant role in image processing, particularly
 in edge detection and feature selection. These techniques enable precise analysis and interpretation of 
 medical images, which is critical for diagnostic and research purposes.
