% ceci est ce que vous voyez de base sur le template, effacer le fichier et enlever l'input pour commencer.

\newpage
\section{Readme}

\letstart{C}{eci} est un template inofficiel de polytech BXL.
Les fichiers pour modifier les imports, style et macros sont dans le dossier config.
Pour modifier la page de garde, veuillez utiliser la section définition à partir de la \textit{ligne} 11 du fichier main.tex.
\\ \\
Si vous ne voulez pas utiliser de chapitres, vous pouvez changer la \textit{ligne} 14 pour changer le nom dans le header par autre chose. par exemple si on veut voir le nom des sections: \\ 
\begin{lstlisting}[style=latex]
\fancyhead[L]{\textit{\rightmark}}
\end{lstlisting}

\noindent
De plus pour enlever la numérotation des chapitres sur les titres de section, rajoutez la commande suivante juste après le $\backslash$begin\{document\}
\begin{lstlisting}[style=latex]
\counterwithout{section}{chapter}
\end{lstlisting}

\subsubsection{Conseil}
\noindent
Faites un dossier par chapitre et un fichier par section, chaque chapitre devrait contenir au moins un dossier image et un fichier "[nom du chapitre]\_input.tex" qui \textit{include} tous les fichiers de section. C'est le seul fichier de ce chapitre que vous devrez mettre dans main.tex et vous pourrez organiser plus facilement les sections.

\subsubsection{Macros déjà présentes}
\noindent
cette macro sert à mettre une lettrine avec des paramètres de base dessus (c'est le grand C au tout début), à utiliser pour avoir un début de chapitre avec du style
\begin{lstlisting}[style=latex]
\letstart{[1er lettre]}{[suite du mot]}
\end{lstlisting}