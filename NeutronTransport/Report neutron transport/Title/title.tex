\begin{titlepage}

% définie les barres au milieux
\newcommand{\HRule}{\rule{\linewidth}{0.5mm}}

% logo polytech
\begin{figure}[H]
    \hspace{-1.5cm}
    \vspace{2cm}
    \includegraphics[width=8cm]{Title/Pictures/logo-polytech.jpg}\\[1cm]
\end{figure}

\center 
%----------------------------------------------------------------------------------------
%	HEADING SECTIONS
%----------------------------------------------------------------------------------------


\textsc{\LARGE ~~\overtitle{}~~}\\[1.5cm] % Nom du cours / matières
\textsc{\Large Université Libre de Bruxelle}\\[0.5cm] % nom de l'université
\textsc{\large Ecole Polytechnique de Bruxelles}\\[0.5cm] % plus précisiment l'école

%----------------------------------------------------------------------------------------
%	TITLE SECTION
%----------------------------------------------------------------------------------------

\makeatletter
\HRule \\[0.4cm]
{ \huge \bfseries \@title}\\[0.4cm] % Title of your document
\HRule \\[1.5cm]
 
%----------------------------------------------------------------------------------------
%	AUTHOR SECTION
%----------------------------------------------------------------------------------------

% partie des auteurs
\begin{minipage}{0.4\textwidth}
\begin{flushleft} \large
\emph{Auteurs:}\\
\@author
\end{flushleft}
\end{minipage}
~ % partie des infos
\begin{minipage}{0.4\textwidth}
    \begin{flushright} \large
        \titleinfo{}
    \end{flushright}
\end{minipage}\\[2cm]
\makeatother

% la date
{\large \docdate{}}\\[2cm]

% définie l'image de fond, tampon ulb
\BgThispage
\backgroundsetup{scale=1,
color=black,
opacity=0.1,
angle=10,
position={12cm,-22cm},contents={%
\includegraphics[height=20cm,width=20cm,keepaspectratio]{Title/Pictures/sceauULB.jpg}}%
}

\vfill
\end{titlepage}
\backgroundsetup{contents={}}